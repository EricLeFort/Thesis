\chapter{Introduction}
Picking the best machine learning algorithm for a problem is of paramount importance; choosing the correct one can be the difference between the success and failure of a project. The goal of this research is to better define the differences between eight classical machine learning techniques. This knowledge can then assist machine learning practitioners in making their decision. The task used to test these techniques is using Ontario Universities' Application Centre (OUAC) application data to predict which applicants will accept an offer of admission to a particular university. The algorithms that will be analyzed are k-nearest neighbours, decision trees, random forests, gradient tree boosting, logistic regression, naive bayes, support vector machines, and artificial neural networks. The performance metrics that will be measured are the final model's accuracy, training time and execution time. This list of algorithms is by no means exhaustive of all known machine learning algorithms. This subset was selected since they are some of the most commonly used for data of this nature.

In addition to the information regarding performance metrics, this paper will also act as a consolidated resource of the mathematical specifications of the algorithms discussed. This analysis takes advantage of an excellent machine learning library written for Python -- Scikit-Learn \cite{scikit-learn}. This library is commonly used among practitioners which ensures the relevancy of the results herein.