\chapter{Introduction}
Selecting the best machine learning algorithm for a problem is of paramount importance; choosing the correct one can be the difference between the success and failure of a project. The goal of this research is to better define the approach to take when inspecting the differences between machine learning techniques as applied to a particular task. This knowledge can then assist machine learning practitioners in making their decision. The task used to test these techniques is using Ontario Universities' Application Centre (OUAC) application data to predict the likelihood of an applicant accepting an offer of admission to a particular university. The algorithms that will be analyzed are k-nearest neighbours, decision trees, random forests, gradient tree boosting, logistic regression, naive bayes, support vector machines, and artificial neural networks. The performance metrics that will be measured are the final model's accuracy, training time and execution time. This list of algorithms is by no means exhaustive of all known machine learning algorithms. This subset was selected since they are some of the most commonly used for data of this nature.

In addition to the information regarding performance metrics, this paper will also act as a consolidated resource of the mathematical specifications of the algorithms discussed. This analysis takes advantage of an excellent machine learning library written for Python -- Scikit-Learn \cite{scikit-learn}. This library is commonly used among practitioners which ensures the relevancy of the results herein.



\section{Definition of the Problem}
The research discussed in this paper revolves around a real-world prediction task which involves predicting the likelihood that an applicant will accept an offer of admission to an undergraduate engineering program at a specific university given their application data. With this information, the admissions department will be able to better judge how many offers they should issue in order to prevent under- or over-enrolment.



\section{Prior Work}
Two fairly recent studies have also analyzed the results of leveraging machine learning to address a similar problem.

The first, Offer Acceptance Prediction of Academic Placement \cite{OfferAcceptancePrediction}, predicts which international applicants to both undergraduate and graduate level programs will accept offers of admission. This study was conducted by Macquarie University in Australia. The set of features used by their study were similar to those used in this study. Some of the notable missing features include: course level grades, the location of the applicants' current residences, the previous school attended, and their ordered choice of university. This study employed most of machine learning techniques used in this study except for gradient tree boosting. This study found artificial neural networks to perform best with a classification accuracy of about 67\%.

The second, Student Yield Maximization Using Genetic Algorithm on a Predictive Enrolment Neural Network Model \cite{StudentYieldMaximization}, addresses a similar problem for undergraduate applications to Southern Illinois University with a focus on distributing scholarships to maximize the number of acceptances. The features used in this study were American College Testing (ACT) scores, GPA/class-rank, expected family contributions, Free Application for Federal Student Aid (FAFSA), zip codes, and scholarship award amounts. The learning algorithm used for the acceptance prediction component of this study were artificial neural networks. A genetic algorithm was then used to learn to allocate scholarship funds in order to maximize the number of acceptances which, although interesting, is not relevant to this paper. This study found that their artificial neural network was able to achieve a classification accuracy of 80\%.